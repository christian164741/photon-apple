\cleardoublepage
%\appendix
\renewcommand{\thechapter}{B}

\renewcommand{\thesection}{\Alph{chapter}.\arabic{section}}



\chapter{Boxenverzeichnis}
\label{anhangB}


%\chapter{Boxenverzeichnis}
\label{chap:boxenverzeichnis}
\thispagestyle{empty}


%\addcontentsline{toc}{chapter}{Anhang B – Boxenverzeichnis}

\section{Einführung}
\vspace{1em}
\begin{tcolorbox}[title=physikalische Boxen, physikbox]
\begin{itemize}
	\item \emph{Albert Einstein, (1909)}\dotfill \pageref{box:einstein1909}
	\item \emph{ Niels Bohr (1933)}\dotfill  \pageref{box:bohr1933}
	\item \emph{Richard P. Feynman, (1965)}\dotfill \pageref{box:feynman1965}
	\item \emph{Das einzelne Photon – wenn es klickt}\dotfill\pageref{box:einzelphoton}
	\item \emph{Wie entstehen verschränkte Photonen?}\dotfill \pageref{box:spdc}
\end{itemize}
\end{tcolorbox}

\vspace{1em}
\begin{tcolorbox}[title=didaktische Boxen,didaktikbox]
\begin{itemize}
		\item \emph{Quantenobjekt statt Lichtkugel} \dotfill\pageref{box:lichtkugel}
	\item \emph{Was ist Realität in der Quantenphysik?} \dotfill\pageref{box:realitaet}
	\item \emph{Was bedeutet Verschränkung?} \dotfill\pageref{box:verschr}
	\item \emph{Was uns die Experimente lehren} \dotfill\pageref{box:experimente}
	\item \emph{Was wir aus Kapitel I mitnehmen} \dotfill\pageref{box:kapitel1faz}
\end{itemize}
\end{tcolorbox}
\section{Der Weg zum Lichtquant}
\vspace{1em}
\begin{tcolorbox}[title=physikalische Boxen,physikbox]
\begin{itemize}
	\item \emph{Isaac Newton (1704), Teilchentheorie} \dotfill\pageref{box:newton}
	\item \emph{Huygens (1690), über Lichtausbreitung}\dotfill\pageref{box:huygens}
	\item \emph{Maxwell (1873)über Licht und \newline elektromagnetische Wellen} \dotfill\pageref{box:maxwell}
	\item \emph{Was ist ein schwarzer Körper?} \dotfill\pageref{box:schwarzerkoerper}
	\item \emph{Einstein (1905), Lichtquant}\dotfill\pageref{box:einstein-lichtquant}
	\item \emph{Robert A. Millikan über Einstein (1916)} \dotfill\pageref{box:millikan-einstein}
	\item \emph{Max Planck – Wissenschaftliche Selbstbiographie}\dotfill\pageref{box:planck-zitat}
\end{itemize}
\end{tcolorbox}

\vspace{1em}
\begin{tcolorbox}[title=didaktische Boxen,didaktikbox]
\begin{itemize}
	\item \emph{Warum versagt die klassische Theorie?} \dotfill\pageref{box:klassik-versagt}
	\item \emph{Wichtige geschichtliche Erkenntnis} \dotfill\pageref{box:geschichte-planck}
\end{itemize}
\end{tcolorbox}

\vspace{1em}
\begin{tcolorbox}[title=mathematische Boxen,mathebox]
\begin{itemize}
	\item \emph{Plancks Strahlungsgesetz: Eine mathematische\newline Interpolation}\dotfill\pageref{box:planck-interpolation}
\end{itemize}
\end{tcolorbox}

\vspace{1em}
\begin{tcolorbox}[title=hypothetische Boxen,hypobox]
\begin{itemize}
	\item \emph{Was wäre, wenn nicht gequantelt wäre?} \dotfill\pageref{box:hypo-keine-quanten}
\end{itemize}
\end{tcolorbox}


\section{Eigenschaften des Photons}
\vspace{1em}
\begin{tcolorbox}[title=physikalische Boxen,physikbox]

\begin{itemize}
	\item \emph{Max Plank (1905)} \dotfill\pageref{box:planck1948}
	\item \emph{Mikrowellenstrahlung}\dotfill\pageref{box:Mikrowellenstrahlung}
	\item \emph{Grünes Licht} \dotfill\pageref{box:grünesLicht}
	\item \emph{Röntgenstrahlen} \dotfill\pageref{box:röntgenstrahlen}
	\item \emph{Photonenimpuls} \dotfill\pageref{box:Photonenimpuls}
	\item \emph{Ionisierende und Nicht-ionisierende Strahlung} \dotfill\pageref{box:ionisierende}
	\item \emph{Hinweis zur Gefährdung} \dotfill\pageref{box:Hinweis zur Gefärdung}
	\item \emph{Fazit zum elektromagnetischen Spektrum} \dotfill\pageref{box:Fazit zum elektro}
	\item \emph{Warum sieht man keinen Unterschied?} \dotfill\pageref{box:Warum sieht man}
    \item \emph{Eigenschaften des Photon-Spins} \dotfill\pageref{box:Eigenschaften des}
	\item \emph{Kommentar zur Darstellung}\dotfill\pageref{box:Kommentar zur Darstellung}
	\item \emph{Didaktischer Merksatz} \dotfill\pageref{box:didaktischerMerksatz}
	\item \emph{Superposition und Polarisation} \dotfill\pageref{box:Superposition}
	\item \emph{Superposition und Polarisation}\dotfill\pageref{box:Superposition und Polarisation}
	\item \emph{Was uns die Polarisation über Photonen verrät} \dotfill\pageref{box:Was uns die}

\end{itemize}
\end{tcolorbox}

\vspace{1em}
\begin{tcolorbox}[title=mathematische Boxen,mathebox]
\begin{itemize}
	\item \emph{Photonen als Energiequanten}\dotfill\pageref{box:Photon als Energiequanten}
\end{itemize}
\end{tcolorbox}

\vspace{1em}
\begin{tcolorbox}[title=Hinweisboxen,hinweisbox]
\begin{itemize}
	\item \emph{Reales Bildmaterial} \dotfill\pageref{keybox:RealesBildmaterial}
	\item \emph{Reales Bildmaterial zur optischen Pinzette} \dotfill\pageref{box:Manipulation kleiner Partikel}
	\item \emph{Fazit zum elektromagnetischen Spektrum} \dotfill\pageref{box:Fazit zum elektro}
	\item \emph{Hinweis zur Grafik: Warum sieht man \newline keinen Unterschied?} \dotfill\pageref{box:Warum sieht man}
\end{itemize}
\end{tcolorbox}



\vspace{1em}
\begin{tcolorbox}[title=hypothetische Boxen,hypobox]
\begin{itemize}
	\item \emph{Was wäre wenn, wenn das Photon\newline eine Masse hätte?} \dotfill\pageref{box:was wäre wenn}
\end{itemize}
\end{tcolorbox}


\section{Experimentelle Bestätigung des Photons}


\vspace{1em}
\begin{tcolorbox}[title=physikalische Boxen,physikbox]
\begin{itemize}
	\item \emph{Philipp Lenard (1902)}\dotfill\pageref{box:Philipp Lenhard}
	\item \emph{Albert Einstein (1905)} \dotfill\pageref{die Erzeuguung von Licht}
	\item \emph{Robert A. Millikan (1916)} \dotfill\pageref{box:Robert A, Millikan}
	\item \emph{Albert Einstein (1905)}\dotfill\pageref{die Erscheinung der Wärm}
	\item \emph{Robert A. Millikan (1916)} \dotfill\pageref{box:einsteins gleichung passt}
	\item \emph{Was ist die Austrittsarbeit \( A \)?} \dotfill\pageref{bos:was ist Austrittsarbeit}
	\item \emph{Was die Photonengrafik zeigen soll} \dotfill\pageref{box:was die photonengrafik}
	\item \emph{Was Antibunching zeigt} \dotfill\pageref{box:wasAntibunching}
	\item \emph{Was der HOM-Effekt zeigt} \dotfill\pageref{box:HOM-Effekt}
\end{itemize}
\end{tcolorbox}

\vspace{1em}
\begin{tcolorbox}[title=mathematische Boxen,mathebox]

\begin{itemize}
	\item \emph{Compton-Formel} \dotfill\pageref{box:comptonFormel}
\end{itemize}
\end{tcolorbox}

\vspace{1em}
\begin{tcolorbox}[title=didaktische Boxen,didaktikbox]
\begin{itemize}
	\item \emph{Didaktische Klarstellung}\dotfill\pageref{box:didaktischeKlarstellung}
	\item \emph{Wellen- als auch Teilcheneigenschaften} \dotfill\pageref{box:wellen}
	\item \emph{Fazit: Ein scheinbar paradoxes Verhalten} \dotfill\pageref{box:Fazit ein scheinbarer}
\end{itemize}
\end{tcolorbox}

\vspace{1em}
\begin{tcolorbox}[title=Hinweisboxen,hinweisbox]
\begin{itemize}
	\item \emph{Fazit} \dotfill\pageref{box:fazit der photo}
	\item \emph{Was diese Darstellung zeigt} \dotfill\pageref{box:was diese Darstellun}
	\item \emph{Was die Experimente über Licht zeigen} \dotfill\pageref{box:was die Experimente}
\end{itemize}
\end{tcolorbox}

\vspace{1em}
\begin{tcolorbox}[title=hypothetisch Boxen,hypobox]
\begin{itemize}
	\item \emph{Schlüsselidee}\dotfill\pageref{box:schlüsselidee}
\end{itemize}
\end{tcolorbox}




\section{Das Photon in der Quantenelektrodynamik}


\vspace{1em}
\begin{tcolorbox}[title=physikalische Boxen, physikbox]
	\begin{itemize}
		\item \textbf{Was ist Quantenelektrodynamik?} \dotfill \pageref{box:was ist quantenelektro}
		\item \textbf{Was ist der Feldformalismus?} \dotfill \pageref{box:was ist Feldformalismus}
		\item \textbf{Was bedeutet Eichsymmetrie?} \dotfill \pageref{box:was bedeutet Eichsy}
		\item \textbf{Folgen der Eichsymmetrie} \dotfill \pageref{box:folgen der Eichsy}
		\item \textbf{Virtuelle Teilchen im Quantenvakuum} \dotfill \pageref{box:virtuelle-teilchen}
		\item \textbf{Virtuelle Photonen als Kraftvermittler} \dotfill \pageref{box:Virtuelle Photonen als kraftvermittler}
		\item \textbf{Was ein Feynman-Diagramm wirklich zeigt} \dotfill \pageref{box:Was ein Feynman-Diagramm}
		\item \textbf{Virtuelles Photon} \dotfill \pageref{box:virtuelles Photon}
		\item \textbf{Reale Photonen} \dotfill \pageref{box:Reale Photonen}
		\item \textbf{Ohne virtuelle Photonen keine QED} \dotfill \pageref{box:Ohne virtuelle Photonen keine}
		\item \textbf{Feldtheorie statt Teilchenmechanik} \dotfill \pageref{box:Feldtheorie statt Teilchenmechanik}
		\item \textbf{Photonenfeld aus der Lagrangedichte} \dotfill \pageref{box:Photonenfeld aus der Lagrangedichte}
		\item \textbf{Was ist ein Photon in der QED?} \dotfill \pageref{box:Warum ist ein Photon in der QED}
		\item \textbf{Physikalische Bedeutung} \dotfill \pageref{box:physikalische Bedeutung}
	\end{itemize}
\end{tcolorbox}

\vspace{1em}

\begin{tcolorbox}[title=mathebox, mathebox]
	\begin{itemize}
		%\item \textbf{Feynman-Regeln einfach erklärt} \dotfill \pageref{box:Feynman-Regeln einfach erklärt}
		\item \textbf{Aufbau der QED-Lagrangedichte} \dotfill \pageref{box:Sufbau der QED-Langrangedichte}
		\item \textbf{Kopplung aus Prinzip} \dotfill \pageref{box:Kopplung aus Prinzip}
		\item \textbf{Feynman-Regeln der QED (vereinfacht)} \dotfill \pageref{box:Feynman-Regeln der QED}
	\end{itemize}
\end{tcolorbox}

\vspace{1em}
\begin{tcolorbox}[title=didaktische Boxen, didaktikbox]
	\begin{itemize}
		\item \textbf{Virtuell heißt nicht: weniger real?} \dotfill \pageref{box:virtuell-denkfehler}
		\item \textbf{Was zeigt das Feynman-Diagramm wirklich?} \dotfill \pageref{box:Was zeigt das Feynman-Diagramm wirklich}
			\item \textbf{Keine „unsichtbare Kraft“ mehr nötig} \dotfill \pageref{box:unsichtbare Kraft}
		\item \textbf{Diagramm ist nicht gleich Realität} \dotfill \pageref{boxx:Diagramm ist nicht gleich realität}
		\item \textbf{Reale oder virtuelle Photonen \newline im Diagramm} \dotfill \pageref{box:Reale oder virtuelle Photonen}
		\item \textbf{Die Quantenelektrodynamik \newline als Erfolgsmodell} \dotfill \pageref{box:Die Quantenelekrodynamik}
		
		
		\item \textbf{Keine Teilchenbahn im Diagramm} \dotfill \pageref{box: Keine Teilchenbahn im Diagramm}
		
		\item \textbf{Warum nicht einfach klassisch?} \dotfill \pageref{box:Warum nicht einfach klassisch?}
		\item \textbf{Vom Lagrange-Term zum Feynman-Vertex} \dotfill \pageref{box:Vom Lagrange-Term zum Feynmann-Vertax}
		
		\item \textbf{Die Kraft entsteht aus dem Ableiter} \dotfill \pageref{box:Die Kraft entsteht aus dem Ableiter}
		\item \textbf{Warum die Regeln funktionieren} \dotfill \pageref{box:Warum die Regeln funktionieren}
		
		\item \textbf{Warum hat das Photon keinen \newline Spin-3-Zustand?} \dotfill \pageref{box:Warum hat das Photon keinen  Spin-3-Zustand}
        	\item \textbf{Warum ist das wichtig} \dotfill \pageref{box:Warum ist wichtig}
	\end{itemize}
\end{tcolorbox}

\vspace{1em}
\begin{tcolorbox}[title=Hinweisboxen, hinweisbox]
	\begin{itemize}
		\item \textbf{On-shell-Bedingung} \dotfill \pageref{box:On-shell-Bedingung}
		\item \textbf{Indirekte Nachweise virtueller Photonen} \dotfill \pageref{box:Nachweis virtueller Photonen}
		\item \textbf{Schleifendiagramme und Präzisionseffekte} \dotfill \pageref{box:Schleifendiagramme}
		\item \textbf{Hinweis für Leser:innen} \dotfill \pageref{box:Hinweis füe Leser}
		\item \textbf{Ausblick auf Kapitel VI} \dotfill \pageref{box:Ausblick auf Kapitel 6}
	\end{itemize}
\end{tcolorbox}
\vspace{1em}

\begin{tcolorbox}[title=hypothetische Boxen, hypobox]
	\begin{itemize}
		\item \textbf{Was wäre, wenn Licht nicht isotrop wäre?} \dotfill \pageref{box:was wäre nicht isotop}
	\end{itemize}
\end{tcolorbox}

\vspace{1em}
\begin{tcolorbox}[title=Warnhinweise, warnbox]
	\begin{itemize}
		\item \textbf{Feynman-Diagramme nicht wörtlich\newline nehmen!} \dotfill \pageref{box:Warnung}
	\end{itemize}
\end{tcolorbox}

\section{Anwendungen des Photons}
\vspace{1em}

\begin{tcolorbox}[title=physikalische Boxen, physikbox]
	\begin{itemize}
		\item \emph{Stimulierte Emission als Grundlage des Lasers} \dotfill\pageref{box:grundlagedeslaser}
		\item \emph{Physikalische Begriffe} \dotfill\pageref{box:begriffe}
		\item \emph{Photonen bei der PET} \dotfill\pageref{box:PET}
		\item \emph{Totalreflexion in Glasfasern} \dotfill\pageref{box:glasfaser}
		\item \emph{Warum kommen manche Photonen \newline nicht auf der Erde an?} \dotfill\pageref{box:photonen auf erde}
		\item \emph{Spektrallinien und Rotverschiebung} \dotfill\pageref{box:spektrallinien}
		\item \emph{Photonen als Messwerkzeug für Raumkrümmung}\dotfill\pageref{box:messwerkzeug}
	\end{itemize}
\end{tcolorbox}



\begin{tcolorbox}[title=didaktische Boxen, didaktikbox]
	\begin{itemize}
		\item \emph{Typenvielfalt von Lasern – ein Überblick} \dotfill\pageref{box:Typenvielfalt von Lasern}
		\item \emph{Anwendungen von Lasern  – \newline Technik (Überblick)} \dotfill\pageref{box:lasertechnik}
		\item \emph{Anwendungen von Lasern  – \newline Forschung (Überblick)}\dotfill\pageref{box:laser-app-forschung}
		\item \emph{Begriffserklärung: Avalanche-Lawine} \dotfill\pageref{box:avalanche}
		\item \emph{Begriffserklärung: Szintillator} \dotfill\pageref{box:szintillator}
		\item \emph{Was zählt als Photon?} \dotfill\pageref{box:photonenzaehlung}
		\item \emph{Begriffserklärung: Annihilation} \dotfill\pageref{box:annihilation}
		\item \emph{Begriffserklärung: Radiopharmakon} \dotfill\pageref{box:radiopharmakon}
		\item \emph{Vorteil optischer Verfahren} \dotfill\pageref{box:optisches Verfahren}
		\item \emph{Didaktische Erläuterung: Interferenz in der OCT} \dotfill\pageref{box:interferenz_oct}
		\item \emph{Was macht QKD sicher?} \dotfill\pageref{box:qkd}
		\item \emph{Wie funktioniert QKD (z.\,B. BB84)?} \dotfill\pageref{box:wie funktioniert QKD}
		\item \emph{Was zeigt ein Spektrum?} \dotfill\pageref{box:was zeigt spektrum}
		\item \emph{Wie funktioniert ein Interferometer?}\dotfill\pageref{box:interferometer}
		\item \emph{Warum sind Photonen so genau messbar?} \dotfill\pageref{box:photonen_genau}
	\end{itemize}
\end{tcolorbox}
\vspace{1em}
\begin{tcolorbox}[title=Hinweis-Boxen, hinweisbox]
	\begin{itemize}
		\item \emph{Didaktischer Vergleich: SPAD vs. PMT} \dotfill\pageref{box:vergleich SPAD}
		\item \emph{Hinweis zur Bedeutung der Detektionstechnologie} \dotfill\pageref{box:detektionstechnologie}
		\item \emph{Laserparameter}\dotfill\pageref{box:laserparameter}
		\item \emph{Zukunft der photonischen Kommunikation} \dotfill\pageref{box:Zukunft Kommunikation}
	\end{itemize}
\end{tcolorbox}




\section{Photonen und die Zukunft der Physik}
\vspace{1em}
\begin{tcolorbox}[title=physikalische Boxen, physikbox]
	\begin{itemize}
		\item \emph{Was macht ein Photon zum Informationsträger?} \dotfill\pageref{box:photon_information}
		\item \emph{Das Hong–Ou–Mandel-Dip-Phänomen} \dotfill\pageref{box:hong_ou_mandel}
		\item \emph{Kernprinzip der Quantenkryptographie} \dotfill\pageref{box:qcrypto_prinzip}
		\item \emph{Photonische Schaltungen vs. Elektronische \newline Schaltungen} \dotfill\pageref{box:photon_vs_electron}
		\item \emph{Warum Photonen für Logikschaltungen \newline interessant sind}\dotfill\pageref{box:optlogik_vorteile}
		\item \emph{Photonen als Boten der Naturgesetze} \dotfill\pageref{box:photonen_grundlagen}
	\end{itemize}
\end{tcolorbox}

\vspace{1em}
\begin{tcolorbox}[title=didaktische Boxen, didaktikbox]
	\begin{itemize}
		\item \emph{Von Elektronik zu Photonik} \dotfill\pageref{box:optlogik_didaktik}
		\item \emph{Photonisches AND-Gatter im \newline Mach--Zehnder-Interferometer} \dotfill\pageref{box:mzi_and}
	\end{itemize}
\end{tcolorbox}

\vspace{1em}
\begin{tcolorbox}[title=hypothetische Boxen, hypobox]
	\begin{itemize}
		\item \emph{Was wäre, wenn optische Computer\newline  Elektronik ablösen?}\dotfill\pageref{box:optlogik_zukunft}
		\item \emph{Was wäre, wenn das Photon nicht das einzige \newline masselose Boson wäre?} \dotfill\pageref{box:photon_neue_physik}
		\item \emph{Was wäre, wenn wie Photonen völlig \newline kontrollieren könnten?} \dotfill\pageref{box:hypo_kapVII}
	\end{itemize}
\end{tcolorbox}

\vspace{1em}
\begin{tcolorbox}[title=Hinweisboxen, hinweisbox]
	\begin{itemize}
		\item \emph{Was bedeutet „Indistinguishability“?} \dotfill\pageref{box:indistinguishability}
		\item \emph{Was bedeutet „Photonik“?} \dotfill\pageref{box:photonics_definition}
	\end{itemize}
\end{tcolorbox}
\section{Das Photon im Standardmodell der Teilchenphysik}
\vspace{1em}

\begin{tcolorbox}[title=physikalische Boxen, physikbox]
	\begin{itemize}
		\item \emph{Masseloses Photon und Spinwirkung über große Distanzen}\dotfill\pageref{box:photon_spin_reichweite}
	\end{itemize}
\end{tcolorbox}

\vspace{1em}
\begin{tcolorbox}[title=didaktische Boxen, didaktikbox]
	\begin{itemize}
		\item \emph{U(1) anschaulich erklärt} \dotfill\pageref{box:u1_kreis}
		\item \emph{Vom \(W^3\) und \(B\) zum Photon} \dotfill\pageref{box:weinberg_mischung}
		\item \emph{Didaktischer Abschluss: Das Photon \newline im Standartmodell} \dotfill\pageref{box:didaktik_kapVIII}
	\end{itemize}
\end{tcolorbox}

\vspace{1em}
\begin{tcolorbox}[title=Hinweisboxen, hinweisbox]
	\begin{itemize}
		\item \emph{Masselosigkeit und Reichweite} \dotfill\pageref{box:reichweite_masselos}
		\item \emph{Vergleich der Eichbosonen} \dotfill\pageref{box:eichbosonen_vergleich}
			%\item \emph{Merksatz zum Photon} – S.~\pageref{box:Merksatz zum Photon}
	\end{itemize}


\end{tcolorbox}
\begin{tcolorbox}[title=hypothetische Boxen, hypobox]
	\begin{itemize}
		
		\item \emph{Was wäre, wenn das Standardmodell\newline nur ein Zwischenschritt wäre?} \dotfill\pageref{Merksatz zum Photon}
	\end{itemize}
\end{tcolorbox}
\section{Anhang-C}
\vspace{1em}
\begin{tcolorbox}[title=Leitgedanke, didaktikbox]
	\begin{itemize}
		\item \emph{Leitgedanke} \dotfill\pageref{box:leitgedanke}
		
	\end{itemize}
\end{tcolorbox}