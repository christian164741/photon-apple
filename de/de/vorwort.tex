cleardoublepage
\thispagestyle{empty}
\pagestyle{fancy}
\fancyfoot[C]{\thepage}   % Zentrierte Seitenzahl unten
\fancyfoot[L]{}
\fancyfoot[R]{}


\begin{center}


{\Large\textbf{Photon – Theorie und Anwendungen}}\\[1.2em]
{\large Dipl.-Ing.\,(FH)\,Christian Weilharter}\\[1.2em]
\textcopyright~2025, Christian Weilharter, Traunstein\\[2em]
\end{center}
\begin{flushleft}
	\begin{tabular}{@{}l l}
	
	
		\textbf{ISBN: (Print)} & 978-3-912302-00-4 \\[0.5em]
			\textbf{ISBN: (E-Book)} & 978-3-912302-01-1 \\[0.5em]
		\textbf{Auflage:} & 1.~Auflage 2025 \\[0.5em]
		\textbf{Satz:} & \LaTeX \\[0.5em]
		\textbf{Verlag:} & Eigenverlag Christian Weilharter \\[0.5em]
		\textbf{Druck:} & Amazon KDP (Print-on-Demand) \\[0.5em]

			\textbf{E-Book-Ausgabe:} & Apple Books\\[0.5em]
		
		\textbf{Kontakt:} & \href{mailto:info@mathandphysics.de}{info@mathandphysics.de}\\[0.5em]
		\textbf{Web:} & \href{https://www.mathandphysics.de}{www.mathandphysics.de}\\
		

	\end{tabular}
\end{flushleft}

\vspace{2em}
\noindent
Alle Rechte vorbehalten. Kein Teil dieses Buches darf ohne schriftliche Genehmigung des Autors 
in irgendeiner Form reproduziert, gespeichert oder übertragen werden, 
weder elektronisch, mechanisch, durch Fotokopien, Aufnahmen noch auf andere Weise.
\begin{center}\small Printed in Germany\end{center}

%\cleardoublepage

\chapter*{Vorwort}
\markboth{Vorwort}{Vorwort} % sorgt für die Kopfzeile, optional
% kein addcontentsline → Vorwort erscheint NICHT im Inhaltsverzeichnis



Die Entstehung dieses Buches war von einer tiefen Faszination für das Licht und seinen fundamentalen Vermittler – das Photon – getragen. In der modernen Physik nimmt das Photon eine zentrale Rolle ein: als Teilchen ohne Masse, aber mit Energie und Impuls; als Bote der elektromagnetischen Wechselwirkung; und als Schlüsselfigur der Quantenmechanik.
	\noindent
Mein Ziel war es, dieses vielseitige Konzept in seiner historischen Entwicklung, seiner physikalischen Bedeutung und seinen technischen Anwendungen so darzustellen, dass sowohl interessierte Laien als auch fortgeschrittene Leser einen fundierten Zugang erhalten – ohne unnötige Vereinfachung, aber stets verständlich.
	\noindent
Bei der Konzeption und Ausarbeitung dieses Buches kam auch moderne Technologie zum Einsatz. Zur Formulierung, Strukturierung und Reflexion von Inhalten wurde das KI-System ChatGPT von OpenAI (Stand 2025) unterstützend eingesetzt. Diese Form der Zusammenarbeit ermöglichte es, Gedanken schneller zu konkretisieren, alternative Formulierungen zu prüfen und komplexe Zusammenhänge noch klarer darzustellen.
Alle Inhalte wurden jedoch vom Autor kritisch geprüft, überarbeitet und verantwortet.
	\noindent
Dieses Buch ist somit nicht nur ein Beitrag zur Didaktik der modernen Physik – es ist auch ein Experiment, wie traditionelle Wissenschaft mit modernen Werkzeugen eine neue Form der Klarheit und Zugänglichkeit gewinnen kann.
	\noindent
Ich hoffe, dass die Begeisterung für das Thema auf die Leserinnen und Leser überspringt – so wie sie mich seit Jahren begleitet.



\begin{flushright}
	\textit{Dipl.-Ing.(FH) Christian Weilharter} \\
	\vspace{0.5em}
	[Traunstein, 2025]
\end{flushright}


