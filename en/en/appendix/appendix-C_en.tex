\appendix
\renewcommand{\thesection}{\thechapter.\arabic{section}}
\chapter*{Appendix C}
\addcontentsline{toc}{chapter}{Appendix C \quad AI in Science – Tool, Not Truth}
\markboth{Appendix C}{Appendix C}
\renewcommand{\thechapter}{C}
\renewcommand{\cftsecnumwidth}{3em}
\setcounter{chapter}{3}
\setcounter{section}{0}

\vspace{1em}
\begin{center}
	\LARGE\textbf{AI in Science – Tool, Not Truth}
\end{center}

\label{chap:AI-in-Science}
\thispagestyle{plain}

\subsection*{Motivation}
\tcbset{didaktikbox/.style={colback=yellow!5!white, colframe=yellow!50!black, fonttitle=\bfseries}}
This book was born from the desire to present complex physical concepts
—particularly the photon and its role in modern physics—in a clear and well-founded way. A new tool was used in the process, one that is becoming increasingly relevant in scientific work: \textbf{artificial intelligence}, specifically the language model ChatGPT by OpenAI.

But how can AI be meaningfully used in science without compromising understanding, precision, or responsibility? And how can this use be disclosed without undermining the scientific integrity of the work itself? This appendix offers a transparent look into how this book was developed and advocates for a responsible use of AI—as a tool, not a source of truth.

\subsection*{What AI Can—and Cannot—Do}

AI-based language models like ChatGPT are powerful aids for writing and structuring. They can:
\begin{itemize}
	\item assist in drafting initial versions of text,
	\item smooth out complex explanations,
	\item offer inspiration or propose outlines,
	\item suggest alternative formulations.
\end{itemize}

However, what they \textbf{cannot} do:
\begin{itemize}
	\item \textbf{understand} scientific content in the proper sense,
	\item \textbf{verify} whether a formula is derived correctly,
	\item \textbf{grasp} the meaning of physical concepts,
	\item \textbf{critically assess or classify} scientific sources.
\end{itemize}

Therefore: AI can be a valuable \emph{support}, but it \textbf{cannot and must not replace the scientific process of understanding}. Anyone using AI must still think for themselves—and critically review all results.

\subsection*{How This Book Was Created}

The contents of this book—including its structure, physical explanations, and mathematical derivations—were conceived, researched, and authored by the writer. ChatGPT was used in the following supporting roles:

\begin{itemize}
	\item for \textbf{formulating individual passages}, such as introductions, summaries, or didactic sections,
	\item for \textbf{stylistic review} of technical passages,
	\item for \textbf{developing outlines} in early stages of work,
	\item for reflecting on \textbf{clarity and reader guidance}.
\end{itemize}

What is essential: \textbf{All scientific statements, formulas, and interpretations were reviewed, questioned, revised, or discarded by the author.} No AI was involved in the development of physical arguments or core content.

\subsection*{Ethical Questions and Scientific Responsibility}

The use of AI in scientific work raises important and justified questions:

\begin{itemize}
	\item How much automation is acceptable without blurring authorship?
	\item How should potential errors be handled?
	\item How transparently must AI usage be disclosed?
\end{itemize}

The answer lies in a fundamental principle of scientific integrity: \textbf{responsibility}. Anyone using AI remains responsible for the result—regardless of whether certain formulations were proposed by a model.

In this sense, AI is not an author but a tool. It can accelerate processes but cannot replace what science is fundamentally about: \textbf{critical thinking, careful examination, and methodological work}.

\subsection*{Recommendations for Use in Research}

For researchers, educators, and students, the following principles can guide a constructive use of AI:

\begin{itemize}
	\item Use AI \textbf{consciously and selectively}—for linguistic support, not for argumentation or proof.
	\item \textbf{Verify all content independently}—especially when it involves complex material.
	\item \textbf{Disclose AI usage clearly} when relevant—e.g., in prefaces, appendices, or submission statements.
	\item Do not use AI for \textbf{deception or window dressing}, but as a tool to better express your own ideas.
\end{itemize}

\subsection*{Conclusion: AI as a Tool—But Human Responsibility Remains}

Artificial intelligence is neither a substitute for nor an opponent of human insight. It is a \textbf{tool} that can assist in scientific communication—\textbf{if used consciously, thoughtfully, and responsibly}.

In this sense, this book is also a contribution to a new, enlightened way of working with technology in science. Not because technology can do everything—but because we have learned how to use it wisely.
\vspace{1em}
\begin{tcolorbox}[didaktikbox, title={Guiding Principle}]
	\label{box:guiding_principle}
	\small
	\textbf{AI is only as powerful as the human using it.} \\
	It can help structure, formulate, and vary text—but without critical thinking, subject knowledge, and human responsibility, it remains a tool without purpose.
\end{tcolorbox}
