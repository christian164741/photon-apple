\cleardoublepage
\appendix
\renewcommand{\thechapter}{A}
\renewcommand{\thesection}{\Alph{chapter}.\arabic{section}}


\chapter{Mathematical Background and Derivations}
\label{anhangA}

This appendix provides a formal and mathematical deepening of the physical
concepts discussed in the main text. The aim is to preserve the didactic
readability of the chapters while at the same time giving interested readers
access to the complete derivations.  

The sections are thematically structured according to the central properties
of the photon, including the energy–momentum relation, the mass hypothesis,
helicity, and polarization.\index{Photon!properties} In this way, the appendix builds a bridge between
the intuitive explanations in the main text and the mathematical rigor of
quantum field theory.

%\section{Chapter I}
\section{Energy–Momentum Relation of the Photon}\index{Energy–momentum relation!photon}
\label{anhangA:energie_impuls}

This section formally derives why a photon has the energy
\[
E = h f
\]
and the momentum
\[
p = \frac{h}{\lambda}
\]
Starting from Maxwell’s equations and their wave equation, it is shown via the
Poynting vector\index{Poynting vector} and the energy density of the electromagnetic field that
quantization of the fields leads to discrete energy portions. This derivation
supplements the intuitive presentation in the main text (Chapter~I).

\section{The Electromagnetic Field in the Relativistic Formulation}\index{Electromagnetic field!relativistic formulation}\index{Field strength tensor}\index{Lagrangian density}
\label{anhangA:feldtheorie}

Here we introduce the formal description of the electromagnetic field:

\begin{itemize}
	\item Four-potential $A^\mu = (\phi, \vec{A})$\index{Four-potential}
	\item Field strength tensor $F^{\mu\nu} = \partial^\mu A^\nu - \partial^\nu A^\mu$\index{Field strength tensor}
	\item Antisymmetry property $F^{\mu\nu} = -F^{\nu\mu}$
	\item Lagrangian density\index{Lagrangian density}
	\[
	\mathcal{L} = -\frac{1}{4} F_{\mu\nu}F^{\mu\nu}
	\]
	\item Connection to quantization: photon as the gauge boson of $U(1)$
	electromagnetism\index{Gauge boson!U(1)}
\end{itemize}

This provides the mathematical foundation for the statement in the main text
that photons are “excitations of the electromagnetic field.”

\section{Formal Description of Entangled Photons}\index{Photons!entanglement}\index{Dirac notation}\index{SPDC (spontaneous parametric down-conversion)}
\label{anhangA:verschr}

The experiments on the generation of entangled photon pairs (SPDC) presented
in the main text (Chapter~I) can be formally described in Dirac notation. A
typical entangled polarization state pair is given by:

\[
|\psi\rangle = \frac{1}{\sqrt{2}}\big( |H\rangle_A \otimes |V\rangle_B +
|V\rangle_A \otimes |H\rangle_B \big)
\]

\begin{itemize}
	\item $|H\rangle$: horizontally polarized state
	\item $|V\rangle$: vertically polarized state
	\item Indices $A, B$: the two photons
\end{itemize}

This formalism makes the correlations observed in the experiment transparent
and shows why classical hidden-variable models are not sufficient.
%\section{Chapter II}
\section{Derivation of the Rayleigh–Jeans Law}\index{Rayleigh–Jeans law!derivation}\index{Rayleigh, Lord}\index{Jeans, James}\index{Ultraviolet catastrophe}
\label{anhangA:rayleigh}

The Rayleigh–Jeans law arises when the electromagnetic modes in a cavity are
treated as harmonic oscillators:

\begin{enumerate}
	\item Count the number of standing waves in the cubic volume $V$.
	\item Each mode has two polarization directions.
	\item According to the equipartition principle of classical thermodynamics,
	each degree of freedom in equilibrium contributes the mean energy $kT$.
\end{enumerate}

The number of modes between the frequencies $\nu$ and $\nu + d\nu$ is
\[
g(\nu)\, d\nu = \frac{8\pi V \nu^2}{c^3}\, d\nu.
\]

Multiplied by $kT$, this yields the spectral energy density
\[
u(\nu, T) = \frac{8\pi \nu^2}{c^3}\, kT,
\]
which, in wavelength form, becomes $u(\lambda, T) = \tfrac{8\pi kT}{\lambda^4}$.
The law is accurate at long wavelengths but diverges for $\lambda \to 0$ – the
so-called \emph{ultraviolet catastrophe}.

\section{Wien’s Radiation Law}\index{Wien’s displacement law}\index{Wien, Wilhelm}
\label{anhangA:wien}

In 1896 Wilhelm Wien derived an approximation for blackbody radiation.
His reasoning was based on:

\begin{itemize}
	\item Thermodynamic considerations: adiabatic compression of a cavity shifts
	the radiation spectrum.
	\item Dimensional analysis: the intensity depends on $T$ and $\lambda$ and
	must have the correct units.
\end{itemize}

The result was
\[
u(\lambda, T) = \frac{c_1}{\lambda^5}\,
\exp\!\left(-\frac{c_2}{\lambda T}\right),
\]
with constants $c_1, c_2$, which were only fully understood through Planck’s
approach. Wien’s law is valid in the UV region but fails at long wavelengths.

\section{Derivation of Planck’s Radiation Law}\index{Planck’s radiation law!derivation}\index{Planck, Max}
\label{anhangA:planck}

Planck combined the limiting cases of Rayleigh–Jeans and Wien and introduced
the quantization of energy:

\begin{itemize}
	\item An oscillator can only take on energies $E_n = nh\nu$.
	\item The occupation probability follows the Boltzmann distribution.
\end{itemize}

The mean energy per oscillator is
\[
\langle E \rangle = \frac{h\nu}{e^{h\nu/kT} - 1}.
\]

Multiplied by the mode number
$g(\nu)\, d\nu = \tfrac{8\pi V \nu^2}{c^3}\, d\nu$
this gives the energy density
\[
u(\nu, T) = \frac{8\pi \nu^2}{c^3}\,
\frac{h\nu}{e^{h\nu/kT} - 1}.
\]

This is \textbf{Planck’s radiation law}, which agrees with experiment at all
frequencies.

\section{Mathematical Description of the Photoelectric Effect}\index{Photoelectric effect!mathematical description}
\label{anhangA:photoeffekt}

In the photoelectric effect, an electron in a metal absorbs a photon with
energy $E_\gamma = h\nu$. To release the electron from the metal, the
\emph{work function} $A$ must be overcome. The excess energy goes into the
electron’s kinetic energy:

\[
E_\text{kin} = h\nu - A.
\]

This leads to a \emph{threshold frequency}
\[
\nu_\text{min} = \frac{A}{h},
\]
below which no electrons are emitted – independent of light intensity.
This linear relation between electron energy and light frequency was confirmed
precisely in Millikan’s experiments (1916).\index{Millikan, Robert A.}
%\section{Chapter III}
\section{Photon Momentum}\index{Photon!momentum}\index{Energy–momentum relation}
\label{anhangA:impuls}

The momentum of a photon can be derived from the relativistic
energy–momentum relation. For arbitrary particles,
\[
E^2 = (pc)^2 + (m_0 c^2)^2 ,
\]
where $E$ is the energy, $p$ the momentum, $c$ the speed of light, and $m_0$ the rest mass.

\begin{itemize}
	\item For massless particles ($m_0 = 0$) this equation reduces to
	\[
	E = p c.
	\]
	
	\item For the photon, the quantization condition holds simultaneously:
	\[
	E = h f = \frac{h c}{\lambda}.
	\]
	
	\item Equating both expressions for $E$ immediately yields
	\[
	p = \frac{E}{c} = \frac{h}{\lambda}.
	\]
\end{itemize}

\noindent
Thus, the momentum of a photon is directly linked to its wavelength. This relation is one of the central bridges between the wave and particle description of light.

\section{Energy–Momentum Relation of the Photon}\index{Photon!mass hypothesis}\index{Photon!masslessness}\index{Energy–momentum relation}
\label{anhangA:masse}

For relativistic particles, the general energy–momentum relation is
\[
E^2 = (pc)^2 + (m c^2)^2.
\]

\begin{itemize}
	\item \textbf{Massless photon:}  
	Setting $m=0$ immediately gives
	\[
	E = p c.
	\]
	This is consistent with the relations $E = h f$ and $p = h/\lambda$.
	
	\item \textbf{Hypothetically massive photon:}  
	If the photon had a rest mass $m_\gamma \neq 0$, then
	\[
	E^2 = (p c)^2 + (m_\gamma c^2)^2.
	\]
	Such a photon would always move slower than $c$, and the speed of light would no longer be universally constant.  
	Even minimal deviations from $m=0$ would be revealed in precision experiments.
\end{itemize}

Experimentally, only an upper bound for the photon mass has been established so far. Current limits are
\[
m_\gamma < 10^{-18}\,\text{eV}/c^2,
\]
which effectively means that the photon is considered massless.

\section{Helicity of the Photon}\index{Photon!helicity}\index{Spin}
\label{anhangA:helizitaet}

The photon has spin $s=1$, but due to its masslessness not all three spin projections ($m_s=-1,0,+1$) are physically realizable. 

\begin{itemize}
	\item \textbf{General spin-1 state:}  
	For massive spin-1 particles, three polarization states are possible,
	corresponding to the projections $m_s=-1,0,+1$ onto the direction of motion.
	
	\item \textbf{Massless photon:}  
	Since the photon has no rest mass, there is no rest frame in which the spin orientation can be defined independently of the momentum vector.  
	Mathematically, the gauge invariance of Maxwell’s equations (or of the QED formalism) enforces the vanishing of the longitudinal component ($m_s=0$).
	
	\item \textbf{Helicity states:}  
	Two possible states remain:
	\begin{align*}
		\lambda &= +1 \quad \text{(right-handed, right circular polarization)}\\
		\lambda &= -1 \quad \text{(left-handed, left circular polarization)}
	\end{align*}
	
	These are called the two helicity states of the photon.
\end{itemize}

\noindent
Thus, the photon is a two-state massless boson whose degrees of freedom are fully described by the two possible helicities.

\section{Polarization of the Photon}\index{Photon!polarization}\index{Dirac notation}\index{Jones vector}
\label{anhangA:polarisation}

Polarization describes the transverse oscillation direction of a photon’s electric field. Formally, this degree of freedom can be represented in two ways:

\begin{itemize}
	\item \textbf{Dirac notation:}  
	In quantum mechanics, polarization states are written as basis vectors in a two-dimensional Hilbert space:  
	\[
	|H\rangle = \begin{pmatrix}1 \\ 0\end{pmatrix}, \qquad
	|V\rangle = \begin{pmatrix}0 \\ 1\end{pmatrix},
	\]
	where $|H\rangle$ stands for horizontal and $|V\rangle$ for vertical polarization.  
	Arbitrary polarization states can be expressed as linear combinations:  
	\[
	|\psi\rangle = \alpha |H\rangle + \beta |V\rangle, \quad |\alpha|^2 + |\beta|^2 = 1.
	\]
	
	\item \textbf{Jones vectors:}  
	In classical optics, the same state is described by \emph{Jones vectors}:  
	\[
	\vec{E} = \begin{pmatrix} E_x \\ E_y \end{pmatrix},
	\]
	where $E_x$ and $E_y$ are the complex amplitudes of the electric field components in the $x$- and $y$-directions.  
	Here too, normalizing the intensity corresponds to normalizing the state vector in Dirac notation.
\end{itemize}

\noindent
Both descriptions are equivalent — Dirac notation emphasizes the quantum mechanical state space, while Jones vectors reflect the classical electromagnetic wave optics.  
Linking these representations is a central tool in quantum optics.
%\section{Chapter IV}
\section{Derivation of Einstein’s Photoelectric Equation}
\label{anhangA:photoeffekt}

Einstein’s equation\index{Einstein’s equation}\index{Photoelectric effect}
\[
E_{\text{kin}} = h \nu - A
\]
follows from a simple energy balance between a photon\index{Photon} and an electron.\index{Electron}

\begin{enumerate}
	\item A photon carries the energy
	\[
	E_{\text{photon}} = h \nu,
	\]
	where \( h \) is Planck’s constant\index{Planck’s constant}\index{Planck, Max} and \( \nu \) is the frequency\index{Frequency} of the incident light.
	
	\item To liberate an electron from a metal, the \textbf{work function} \( A \)\index{Work function} must be overcome. This corresponds to the minimal binding energy of electrons in the solid.
	
	\item If energy remains after overcoming \( A \), it appears as the electron’s kinetic energy:\index{Kinetic energy}
	\[
	E_{\text{kin}} = E_{\text{photon}} - A.
	\]
	
	\item Hence, directly:
	\[
	E_{\text{kin}} = h \nu - A.
	\]
\end{enumerate}

\textbf{Remark.}  
If a retarding (stopping) voltage \( U \)\index{Stopping potential} is applied, then
\[
eU = h \nu - A,
\]
where \( e \) is the elementary charge.\index{Elementary charge} This form allows a direct experimental determination of \( h \) by measuring the stopping potential as a function of the frequency.

\section{Planck–Einstein Relation \texorpdfstring{$E = h\nu$}{E = hν}}
\label{anhangA:planckEinstein}
\index{Planck–Einstein relation}

\textbf{Goal.} Justify why a single light quantum (photon) carries the energy
\[
E = h\nu = \hbar \omega .
\]

\subsection*{Route 1: Quantizing the normal modes of the electromagnetic field.}
\phantomsection
The free electromagnetic field\index{Electromagnetic field} in a volume \(V\) can be decomposed into plane normal modes with angular frequencies \(\omega_{\mathbf{k}}\). Each mode is a harmonic oscillator\index{Harmonic oscillator} with Hamiltonian
\[
\hat H_{\mathbf{k}}=\hbar \omega_{\mathbf{k}}\!\left(\hat a_{\mathbf{k}}^\dagger \hat a_{\mathbf{k}}+\tfrac12\right).
\]
The eigenvalues are \((n+\tfrac12)\hbar\omega_{\mathbf{k}}\), \(n\in \mathbb{N}_0\). 
An excitation \(\Delta n = 1\) raises the energy by \(\Delta E = \hbar \omega\). 
We identify this increase with \emph{one photon} in that mode:
\[
E_{\text{photon}}=\hbar\omega=h\nu.
\]

\subsection*{Route 2: From Planck’s quantum hypothesis to Einstein’s light quantum.}
\phantomsection
Planck\index{Planck, Max} postulated in 1900 discrete energies \(E_n=n h\nu\) for matter oscillators. 
Einstein\index{Einstein, Albert} in 1905 applied quantization to the \emph{radiation field} itself: the energy of radiation behaves as if bundled into spatially localized \emph{energy packets} of size \(h\nu\). 
Only then can, among other things, the entropy properties behind Wien’s law\index{Wien’s law} and the photoelectric effect be explained consistently. 
Thus for a single light quantum,
\[
E_{\text{photon}} = h\nu .
\]

\subsection*{Consequences.}
\phantomsection
(i) Photon energy depends \emph{only} on frequency (not on intensity\index{Intensity}). 
(ii) Together with \(E_{\text{kin}}=h\nu-A\) this explains the threshold frequency\index{Threshold frequency} in the photoelectric effect. 
(iii) With field quantization this leads to the number operator \(\hat N=\hat a^\dagger \hat a\)\index{Number operator} and a clear assignment of energy per photon.

\section{Derivation of the Stopping-Potential Equation}
\label{anhangA:stoppspannung}

\textbf{Goal.} Connect photon energy, work function, and the measurable retarding voltage \(U\).

\subsection*{Starting point.}
\phantomsection
The energy balance in the photoelectric effect is
\[
E_{\text{photon}} = h\nu = A + E_{\text{kin,max}}.
\]

\subsection*{Experimental principle.}
\phantomsection
In a photocell,\index{Photocell} a \emph{retarding (stopping) voltage} \(U\)\index{Stopping potential} is applied between cathode\index{Cathode} and anode.\index{Anode}  
Electrons with kinetic energy \(E_{\text{kin}}\) must do work \(eU\) to reach the anode.  
At the \textbf{stopping potential} \(U_0\) the energy is just exhausted:
\[
E_{\text{kin,max}} = eU_0.
\]

\subsection*{Derivation.}
\phantomsection
Inserting into the balance,
\[
h\nu = A + eU_0 \;\Rightarrow\;
eU_0 = h\nu - A.
\]

\subsection*{Experimental significance.}
\phantomsection
– The graph \(U_0(\nu)\) is a straight line with slope \(h/e\).  
– The intercept yields the material-dependent work function \(A\).  
– Millikan\index{Millikan, Robert A.} (1916) determined Planck’s constant with high precision this way, confirming Einstein’s hypothesis.

\subsection*{Consequence.}
\phantomsection
The stopping potential enables a direct measurement of fundamental constants, independent of light intensity or photon number.

\section{The Work Function}
\label{anhangA:austrittsarbeit}

The \textbf{work function} \( A \) is the minimum energy required to free an electron from a metal. 
It depends on the material and the electronic structure of the surface. 

\textbf{Formal definition:}
\[
A = E_{\text{Fermi}} + E_{\text{binding}} - E_{\text{vacuum}},
\]
where \( E_{\text{vacuum}} \) is the energy level of an electron in vacuum.

\textbf{Typical categories:}
\begin{itemize}
	\item Alkali metals (e.g., cesium, potassium)
	\item Transition metals (e.g., iron, copper)
	\item Noble metals (e.g., platinum)
\end{itemize}

The work function explains why only photons above a \emph{threshold frequency} \( \nu_0 = A/h \) can liberate electrons. 
It is material-specific and can vary with surface condition, temperature, or coatings.

\section{Derivation of the Compton Formula}
\label{anhangA:comptonHerleitung}

The derivation of the \textbf{Compton formula}\index{Compton formula} is based on energy and momentum conservation\index{Momentum conservation} in the collision of a photon with a stationary electron. 

\textbf{Setup:}
\begin{itemize}
	\item A photon with wavelength \( \lambda \) hits an electron at rest.
	\item After the collision the photon has wavelength \( \lambda' \) and is scattered by an angle \( \theta \).
	\item The electron acquires a recoil momentum \( \vec{p}_e \).
\end{itemize}

\textbf{Conservation laws:}
\begin{align*}
	E_\gamma + m_e c^2 &= E'_\gamma + E_e ,\\
	\vec{p}_\gamma &= \vec{p}\,'_\gamma + \vec{p}_e ,
\end{align*}
with
\[
E_\gamma = \frac{hc}{\lambda}, \quad 
E'_\gamma = \frac{hc}{\lambda'}, \quad 
p_\gamma = \frac{h}{\lambda}, \quad 
E_e^2 = (p_e c)^2 + (m_e c^2)^2.
\]

\textbf{Result:}
\[
\Delta \lambda = \lambda' - \lambda 
= \frac{h}{m_e c}(1 - \cos \theta).
\]

This shift is independent of photon energy and depends only on the scattering angle. The factor
\[
\lambda_C = \frac{h}{m_e c} \approx 2.43 \times 10^{-12}\,\mathrm{m}
\]
is the \textbf{Compton wavelength} of the electron.\index{Compton wavelength}\index{Compton, Arthur}

\section{The Double-Slit in the \newline Quantum-Mechanical Formalism}
\label{anhangA:doppelspalt}

The double-slit experiment\index{Double-slit experiment} with single photons can only be understood using quantum mechanics.\index{Quantum mechanics} 
Unlike classical wave theory\index{Wave theory} or classical particle mechanics,\index{Classical mechanics} one considers the photon’s \textbf{wavefunction}\index{Wavefunction} and its superposition.

\textbf{Superposition principle.}\index{Superposition principle}  
Given two possible paths \( W_1 \) and \( W_2 \), the total amplitude is
\[
\Psi_{\text{total}} = \Psi_{1} + \Psi_{2}.
\]

\textbf{Dirac-notation representation.}\index{Dirac notation}  
Let \(|1\rangle\) be “photon goes through slit 1” and \(|2\rangle\) be “photon goes through slit 2.”  
Without which-path measurement:
\[
|\psi\rangle = \frac{1}{\sqrt{2}} \left( |1\rangle + |2\rangle \right).
\]

\section{Antibunching and the Second-Order Correlation Function}
\label{anhangA:antibunching}

The phenomenon of \textbf{antibunching}\index{Antibunching} shows that photons are emitted \emph{one by one}. 
Mathematically, this is described by the second-order correlation function.\index{Correlation function}

\textbf{Definition:}
\[
g^{(2)}(\tau) = \frac{\langle I(t) \, I(t+\tau) \rangle}{\langle I(t) \rangle^2},
\]
where \( I(t) \) is the intensity (or count rate) at the detector and \(\tau\) the time delay between two measurements.

\textbf{Antibunching.}
For an ideal single-photon source,\index{Single-photon source}
\[
g^{(2)}(0) = 0.
\]

\textbf{Physical consequences.}
– Antibunching contradicts any classical wave picture.  
– It shows the \textbf{indivisibility of the photon}: it is detected here or there — but never simultaneously at two places.  
– Thus, antibunching is a direct proof of the quantum nature of light.

\section{Hong–Ou–Mandel Interference at a Beam Splitter}
\label{anhangA:HOM}

The \textbf{Hong–Ou–Mandel (HOM) effect}\index{Hong–Ou–Mandel effect}\index{Hong, Chung-ki}\index{Mandel, Leonard} describes two-photon interference\index{Two-photon interference} of identical photons at a 50:50 beam splitter.\index{Beam splitter}
For perfect indistinguishability\index{Indistinguishability of photons} the coincidences at the two outputs vanish (“HOM dip”).\index{HOM dip}

\subsection*{Beam-splitter transformation (Heisenberg picture).}\index{Heisenberg picture}
\phantomsection
For input modes \(\hat a,\hat b\) and output modes \(\hat c,\hat d\) of a lossless 50:50 beam splitter we choose the unitary map
\[
\begin{pmatrix}
	\hat c \\ \hat d
\end{pmatrix}
= \frac{1}{\sqrt{2}}
\begin{pmatrix}
	1 & i \\
	i & 1
\end{pmatrix}
\begin{pmatrix}
	\hat a \\ \hat b
\end{pmatrix},
\qquad
\begin{pmatrix}
	\hat a \\ \hat b
\end{pmatrix}
= \frac{1}{\sqrt{2}}
\begin{pmatrix}
	1 & -i \\
	-i & 1
\end{pmatrix}
\begin{pmatrix}
	\hat c \\ \hat d
\end{pmatrix}.
\]
For the creation operators,
\[
\hat a^\dagger=\frac{\hat c^\dagger - i \hat d^\dagger}{\sqrt{2}},
\qquad
\hat b^\dagger=\frac{-i\,\hat c^\dagger + \hat d^\dagger}{\sqrt{2}}.
\]

\subsection*{Input and output states.}
\phantomsection
Two single photons, one per input mode,
\(|\psi_{\text{in}}\rangle=\hat a^\dagger \hat b^\dagger |0\rangle\),
lead to
\[
|\psi_{\text{out}}\rangle
= \hat a^\dagger \hat b^\dagger |0\rangle
= \frac{1}{2}\,(\hat c^\dagger - i \hat d^\dagger)(-i\,\hat c^\dagger + \hat d^\dagger)\,|0\rangle
= -\frac{i}{2}\!\left(\hat c^{\dagger 2}+\hat d^{\dagger 2}\right)|0\rangle.
\]
Since the \(\hat c^\dagger \hat d^\dagger\) terms cancel exactly, the output state contains \emph{no} \(|1_c,1_d\rangle\) component (no coincidences). After normalization one may write equivalently
\[
|\psi_{\text{out}}\rangle
\propto |2_c,0_d\rangle \,\pm\, |0_c,2_d\rangle,
\]
where the relative sign depends only on the beam-splitter phase convention; the physics (vanishing coincidences) is unchanged.

\subsection*{Imperfect overlap and the HOM dip.}
\phantomsection
Real photons are finite wave packets in time/spectrum/polarization. Let
\(\Lambda(\tau)=\int\!dt\, f_a(t)\,f_b^*(t+\tau)\)
be the (complex) temporal overlap integral (delay \(\tau\)).
Then the coincidence probability at the outputs is
\[
P_{\text{coinc}}(\tau)=\frac{1}{2}\Bigl(1-|\Lambda(\tau)|^2\Bigr).
\]
For perfectly overlapping, indistinguishable photons, \(|\Lambda(0)|=1\Rightarrow P_{\text{coinc}}(0)=0\).
For two Gaussian wave packets with coherence time \(\tau_c\),
\(|\Lambda(\tau)|^2=\exp[-(\tau/\tau_c)^2]\),
so
\[
P_{\text{coinc}}(\tau)=\tfrac{1}{2}\Bigl(1-e^{-(\tau/\tau_c)^2}\Bigr)
\]
shows the characteristic \emph{HOM dip}.

\subsection*{Role of indistinguishability.}
\phantomsection
Any distinguishability (polarization angle \(\Delta\phi\), spectral or spatial mode mismatch) reduces the visibility \(V\in[0,1]\):
\[
P_{\text{coinc}}(\tau)=\frac{1}{2}\Bigl(1- V\,|\Lambda(\tau)|^2\Bigr),
\qquad
V=|\langle \xi_a|\xi_b\rangle|^2,
\]
where \(|\xi_{a,b}\rangle\) collect all \emph{internal} degrees of freedom (e.g., polarization).

\subsection*{Remark.}
\phantomsection
The disappearance of coincidences is not a classical field-interference effect but a \emph{two-photon interference} of probability amplitudes,\index{Probability amplitude} proving indistinguishability\index{Indistinguishability of photons} and the bosonic nature of photons.\index{Bosons}
%\section{Chapter V}
\section{Field Formalism and Four-Potential}
\label{anhangA:feldformalismus}
\label{anhangA:viererpotential} % both labels point to the same spot

In quantum electrodynamics (QED)\index{Quantum electrodynamics (QED)} the photon is not described as a classical particle,
but rather as an excitation of the \emph{electromagnetic field}\index{Electromagnetic field}. 
This field is represented by the \textbf{four-potential} \( A^\mu(x) \)\index{Four-potential},
which in the relativistic formulation comprises four components:

\[
A^\mu(x) = \big( \Phi(x), \, \vec{A}(x) \big) ,
\]

where \( \Phi(x) \)\index{Electric potential} is the electric potential and \( \vec{A}(x) \)\index{Magnetic vector potential} is the magnetic vector potential. 
The temporal and spatial components combine into a Lorentz vector\index{Lorentz vector}.

\subsection*{Field strength tensor.}
\phantomsection
From the four-potential one obtains the \textbf{field strength tensor}\index{Field strength tensor}

\[
F_{\mu\nu} = \partial_\mu A_\nu - \partial_\nu A_\mu ,
\]

which contains the physical fields:
\[
\vec{E} = -\nabla \Phi - \frac{\partial \vec{A}}{\partial t}, 
\quad
\vec{B} = \nabla \times \vec{A}.
\]

\subsection*{Lagrangian density.}
\phantomsection
The dynamics of the electromagnetic field are derived from the
\textbf{Lagrangian density}\index{Lagrangian density}

\[
\mathcal{L}_{\text{EM}} = - \tfrac{1}{4} F_{\mu\nu} F^{\mu\nu} .
\]

Through the principle of least action\index{Principle of least action}, this formulation leads to 
Maxwell’s equations\index{Maxwell's equations}\index{Maxwell, James Clerk} in their relativistic form.

\subsection*{Gauge symmetry.}
\phantomsection
The potential \( A^\mu \) is not uniquely determined:
\[
A^\mu(x) \;\;\rightarrow\;\; A^\mu(x) + \partial^\mu \Lambda(x).
\]
This freedom is called \textbf{gauge symmetry}\index{Gauge symmetry} and guarantees that only
the physically measurable quantities \( \vec{E} \)\index{Electric field} and \( \vec{B} \)\index{Magnetic field}
are independent of the choice of potential.

\medskip
Thus, the four-potential is the central mathematical structure from which
both classical electrodynamics and the quantized form of QED
can be systematically developed.

\section{From the Classical Field to QED}
\label{anhangA:feld_zu_qed}

Classical electrodynamics in the sense of Maxwell describes electric and 
magnetic fields as continuous waves propagating through space\index{Maxwell's equations}\index{Maxwell, James Clerk}. 
In this view, the electromagnetic field is a 
\emph{deterministic solution} of Maxwell’s equations.

\subsection*{Limits of the classical model.}
\phantomsection
Phenomena such as the photoelectric effect\index{Photoelectric effect} and Compton scattering\index{Compton scattering}
demonstrate that light does not interact with matter in arbitrarily divisible amounts, 
but rather in discrete energy packets \( h\nu \). 
This points to an underlying quantum nature of the field.

\subsection*{Quantization of the field.}
\phantomsection
Quantum electrodynamics (QED) goes beyond the classical theory by 
\emph{quantizing} the electromagnetic field itself. Specifically:
\begin{itemize}
	\item The potential \( A^\mu(x) \) becomes an operator field\index{Operator field}.
	\item Its Fourier modes are identified with creation and annihilation operators 
	for photons\index{Creation operator}\index{Annihilation operator}.
	\item Field states are described in Fock space\index{Fock space}, 
	with the possibility of creating arbitrarily many photons in specified modes.
\end{itemize}

\subsection*{New perspective.}
\phantomsection
The photon thus appears as the \textbf{quantum of the electromagnetic field}, 
no longer as a classical particle or wave packet. 
Interactions such as the scattering of two electrons can be understood as 
\emph{photon exchange}, represented mathematically by \textbf{Feynman diagrams}\index{Feynman diagrams}. 

\medskip
In this way, QED bridges classical field theory, quantum mechanics, and special relativity\index{Special relativity}. 
It provides a consistent theoretical foundation in which the photon is described as a 
fundamental exchange particle—a vector boson\index{Vector boson}.

\section{Field Strength Tensor \(F_{\mu\nu}\)}
\label{anhangA:feldstaerketensor}

The foundation of the relativistic formulation of electrodynamics
is the \textbf{field strength tensor} \( F_{\mu\nu} \)\index{Field strength tensor}.
It combines the electric and magnetic fields into a 
covariant form and is directly derived from the four-potential 
\( A^\mu(x) \)\index{Four-potential}:

\[
F_{\mu\nu} \;=\; \partial_\mu A_\nu - \partial_\nu A_\mu .
\]

\subsection*{Properties.}
\phantomsection
\begin{itemize}
	\item \( F_{\mu\nu} \) is \emph{antisymmetric}, i.e. 
	\( F_{\mu\nu} = - F_{\nu\mu} \).  
	\item It contains exactly six independent components, 
	which correspond to the three components of the electric field \( \vec{E} \)\index{Electric field} 
	and the three components of the magnetic field \( \vec{B} \)\index{Magnetic field}.
\end{itemize}

\subsection*{Matrix representation.}
\phantomsection
In 3+1 notation one obtains:

\[
F_{\mu\nu} = 
\begin{pmatrix}
	0      & -E_x & -E_y & -E_z \\
	E_x    & 0    & -B_z & B_y \\
	E_y    & B_z  & 0    & -B_x \\
	E_z    & -B_y & B_x  & 0
\end{pmatrix}.
\]

\subsection*{Lorentz covariance.}
\phantomsection
By its very definition, \( F_{\mu\nu} \) is a rank-two tensor
and transforms consistently under Lorentz transformations\index{Lorentz transformation}.
This guarantees that electric and magnetic fields
are not independent entities, but instead transform into one another 
depending on the observer.
\subsection*{Physical significance.}
\phantomsection
\begin{itemize}
	\item The field strength tensor is the central quantity in the 
	Lagrangian formulation of electrodynamics\index{Electrodynamics}.
	\item It allows for a compact representation of Maxwell’s equations\index{Maxwell's equations}\index{Maxwell, James Clerk}.
	\item In quantum electrodynamics (QED)\index{Quantum electrodynamics (QED)} it provides the basis 
	for defining photon fields\index{Photon field} and their interactions\index{Interaction}.
\end{itemize}

\medskip
Thus, \( F_{\mu\nu} \) unifies the classical fields \( \vec{E} \)\index{Electric field} and \( \vec{B} \)\index{Magnetic field}
into a single relativistically invariant structure.

\section{EM Lagrangian Density and Equations of Motion}
\label{anhangA:lagrange_em}

The dynamics of the electromagnetic field can be elegantly expressed 
through a \textbf{Lagrangian density}\index{Lagrangian density}. 
The starting point is the field strength tensor \( F_{\mu\nu} \) 
(see Section~\ref{anhangA:feldstaerketensor}).

\subsection*{Lagrangian density.}
\phantomsection
The canonical form is
\[
\mathcal{L}_{\text{EM}} \;=\; -\tfrac{1}{4} \, F_{\mu\nu} F^{\mu\nu}.
\]

\begin{itemize}
	\item The prefactor \(-\tfrac{1}{4}\) is necessary to obtain the correct normalization 
	under variation.
	\item The contracted form \(F_{\mu\nu} F^{\mu\nu}\) 
	is a Lorentz scalar\index{Lorentz scalar}, i.e. invariant under Lorentz transformations\index{Lorentz transformation}.
\end{itemize}

\subsection*{Coupling to matter.}
\phantomsection
For the field to interact with charged particles, 
an interaction term is added:
\[
\mathcal{L}_{\text{int}} \;=\; - j_\mu A^\mu ,
\]
where \( j_\mu \)\index{Four-current} is the four-current.

\subsection*{Variation and field equations.}
\phantomsection
Applying the \textbf{principle of least action}\index{Principle of least action} to
\[
\mathcal{L} \;=\; -\tfrac{1}{4} F_{\mu\nu}F^{\mu\nu} - j_\mu A^\mu
\]
and varying with respect to the potential \( A^\mu \), one obtains:
\[
\partial_\nu F^{\mu\nu} \;=\; j^\mu .
\]

These are Maxwell’s equations in compact, 
covariant form. The inhomogeneous equations
(\( \nabla \cdot \vec{E} = \rho, \, \nabla \times \vec{B} - \tfrac{\partial \vec{E}}{\partial t} = \vec{j} \))
are included within them.

\subsection*{Homogeneous equations.}
\phantomsection
The remaining two Maxwell equations 
(\( \nabla \cdot \vec{B} = 0, \, \nabla \times \vec{E} + \tfrac{\partial \vec{B}}{\partial t} = 0 \))
follow from the definition of the field strength tensor
and the identity
\[
\partial_\lambda F_{\mu\nu} + \partial_\mu F_{\nu\lambda} + \partial_\nu F_{\lambda\mu} = 0 .
\]

\medskip
This shows that the entirety of classical electrodynamics 
can be derived from a compact Lagrangian formulation — 
an elegant starting point for quantization within the framework of QED.
\section{Gauge Symmetry, Gauge Fixing,\newline and the  Lorenz Condition}
\label{anhangA:eichsymmetrie}

A central structural principle of electrodynamics is \textbf{gauge symmetry}\index{Gauge symmetry}.
It states that the four-potential \( A^\mu(x) \) is not uniquely determined,
but only up to a \emph{gauge transformation}:
\[
A^\mu(x) \;\;\rightarrow\;\; A^\mu(x) + \partial^\mu \Lambda(x),
\]
where \( \Lambda(x) \) is an arbitrary scalar function.

\subsection*{Physical consequences.}
\phantomsection
\begin{itemize}
	\item The observable fields \( \vec{E} \)\index{Electric field} and \( \vec{B} \)\index{Magnetic field} 
	remain unchanged under this transformation.
	\item Only gauge-invariant quantities are physically measurable.
	\item A mass term \( \tfrac{1}{2} m^2 A_\mu A^\mu \) 
	is not gauge invariant and is therefore excluded in QED\index{Quantum electrodynamics (QED)}.
\end{itemize}

\subsection*{Gauge freedom and degrees of freedom.}
\phantomsection
A vector field \( A^\mu \) has four components.  
Gauge symmetry allows one to remove redundant degrees of freedom:
\begin{itemize}
	\item The gauge transformation removes one component.
	\item The equations of motion (Lorentz invariance\index{Lorentz invariance}) remove another.
	\item Exactly two independent degrees of freedom remain — 
	the two transverse polarization states of the photon.
\end{itemize}

\subsection*{Gauge fixing.}
\phantomsection
For practical calculations one chooses a specific \emph{gauge}:
\begin{itemize}
	\item \textbf{Lorenz gauge:} 
	\(\partial_\mu A^\mu = 0\)\index{Lorenz gauge}\index{Lorenz condition}\index{Lorenz, Ludvig}.  
	It is Lorentz covariant and well-suited to relativistic formulations.
	\item \textbf{Coulomb gauge:} 
	\(\nabla \cdot \vec{A} = 0\)\index{Coulomb gauge}.  
	Used frequently in quantum optics.
\end{itemize}

\subsection*{Lorenz condition.}
\phantomsection
In Lorenz gauge the equations of motion reduce to a wave (d’Alembert) equation:
\[
\square A^\mu(x) = j^\mu(x),
\]
with the d’Alembert operator \(\square = \partial_\mu \partial^\mu\)\index{d'Alembert operator}.
This makes the wave nature of the electromagnetic field explicit.

\medskip
Gauge symmetry is thus not merely a mathematical convenience, 
but the reason the photon is \textbf{massless} and has exactly two transverse polarization states.
\newpage
\noindent
\section{Why the Photon Is Massless \newline (Proca Argument)}
\label{anhangA:masselosigkeit_proca}

In classical field theory one could formally add a mass term for a vector field \( A^\mu \).
The corresponding Lagrangian density (Proca theory) is
\[
\mathcal{L}_{\text{Proca}} = -\tfrac{1}{4} F_{\mu\nu} F^{\mu\nu} 
+ \tfrac{1}{2} m^2 A_\mu A^\mu .
\]\index{Proca theory}\index{Proca, Alexandru}

\subsection*{Consequences of the mass term.}
\phantomsection
\begin{itemize}
	\item The equations of motion are modified and yield a massive wave equation.
	\item A massive spin-1 field has \textbf{three} independent polarization states (not two).
	\item The propagation speed would be less than the speed of light.
\end{itemize}

\subsection*{Gauge symmetry is violated.}
\phantomsection
The mass term \( \tfrac{1}{2} m^2 A_\mu A^\mu \) 
is not invariant under the gauge transformation
\[
A^\mu \;\to\; A^\mu + \partial^\mu \Lambda(x),
\]
thus breaking the fundamental \(U(1)\) gauge symmetry\index{U(1) gauge symmetry} of electrodynamics.

\subsection*{Experimental evidence.}
\phantomsection
All observations show that:
\begin{itemize}
	\item electromagnetic waves always propagate at the speed of light \(c\),
	\item the photon has only two transverse polarization states,
	\item and no deviation from perfect masslessness has been detected.
\end{itemize}
Hence gauge symmetry holds exactly, and the photon has exactly zero rest mass\index{Massless particle}.

\medskip
The \textbf{Proca argument} thus shows:  
Gauge symmetry in QED forbids a mass term and forces the photon to be massless.

\section{Transversality and Helicity \( \pm 1 \)}
\label{anhangA:transversalitaet}

The photon is a massless spin-1 particle. Its physically allowed polarization states
follow from \textbf{gauge symmetry} and \textbf{Lorentz invariance}\index{Lorentz invariance}.

\subsection*{Reduction of degrees of freedom.}
\phantomsection
A vector field \( A^\mu \) starts with four components.
\begin{itemize}
	\item Gauge freedom removes one component (by a gauge choice).
	\item The equations of motion (e.g., the Lorenz condition \( \partial_\mu A^\mu = 0 \)) remove another.
	\item Exactly \textbf{two independent degrees of freedom} remain.
\end{itemize}

\subsection*{Transversality.}
\phantomsection
The two remaining polarization modes are orthogonal to the direction of propagation:
\[
\vec{k} \cdot \vec{\epsilon}_\lambda = 0,
\]
where \( \vec{k} \) is the wave vector and \( \vec{\epsilon}_\lambda \) the polarization vector\index{Transversality}\index{Polarization}.

\subsection*{Helicity.}
\phantomsection
For a massless particle the \textbf{helicity}\index{Helicity} — the projection of spin onto the direction of motion —
\[
h = \frac{\vec{S} \cdot \vec{p}}{|\vec{p}|}
\]
is a well-defined, Lorentz-invariant quantity.  
The photon has exactly two helicity states:
\[
h = +1 \quad \text{and} \quad h = -1 .
\]

\subsection*{Physical interpretation.}
\phantomsection
\begin{itemize}
	\item Helicity \( +1 \): right-circularly polarized photons.
	\item Helicity \( -1 \): left-circularly polarized photons.
\end{itemize}

\medskip
Thus the photon is \textbf{transversely polarized} with only two possible helicities —
a direct consequence of gauge symmetry and masslessness 
(see Section~\ref{anhangA:masselosigkeit_proca}).

\section{Virtual Photons and Unphysical \newline Modes}
\label{anhangA:virtuelle_moden}

Whereas real photons have only two transverse helicity states 
(\( h = \pm 1 \)), \textbf{virtual photons}\index{Virtual photon} in QED can involve additional modes. 
They appear on internal lines of Feynman diagrams and are not observable particles,
but mathematical auxiliaries of the formalism.

\subsection*{Unphysical components.}
\phantomsection
Photon propagators can contain, besides transverse parts,
longitudinal or even scalar components.  
These arise from gauge freedom and are unavoidable when one allows propagating solutions in all components.

\subsection*{Consistency of the theory.}

\phantomsection
Although such unphysical modes appear in intermediate steps,
they cancel out in all \emph{physical observables}.  
This occurs due to:
\begin{itemize}
	\item the gauge invariance of the theory,
	\item coupling only to the conserved electric current \( j^\mu \)\index{Conserved current},
	\item and the Ward identities\index{Ward identities}, which ensure the cancellation of non-transverse contributions.
\end{itemize}

\subsection*{Example: photon propagator.}
\phantomsection
In Feynman gauge the photon propagator is
\[
D_{\mu\nu}(k) = \frac{-i g_{\mu\nu}}{k^2 + i\epsilon} ,
\]\index{Feynman gauge}\index{Photon propagator}
which formally includes longitudinal and time-like parts.  
In physical amplitudes these couple in such a way that they cancel exactly.

\medskip
\textbf{Conclusion:}  
Virtual photons are a computational device in QED.  
They may carry seemingly unphysical modes, but gauge symmetry and current conservation 
guarantee that only the two transverse helicity states of the photon are physically realized.
